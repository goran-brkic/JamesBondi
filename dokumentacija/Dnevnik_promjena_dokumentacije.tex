\chapter{Dnevnik promjena dokumentacije}
	
				
		
		\begin{longtabu} to \textwidth {|X[2, l]|X[13, l]|X[3, l]|X[3, l]|}
			\hline \multicolumn{1}{|l|}{\textbf{Rev.}}	& \multicolumn{1}{l|}{\textbf{Opis promjene/dodatka}} & \multicolumn{1}{|l|}{\textbf{Autori}} & \multicolumn{1}{l|}{\textbf{Datum}} \\[3pt] \hline
			\endfirsthead
			
			\hline \multicolumn{1}{|l|}{\textbf{Rev.}}	& \multicolumn{1}{l|}{\textbf{Opis promjene/dodatka}} & \multicolumn{1}{|l|}{\textbf{Autori}} & \multicolumn{1}{l|}{\textbf{Datum}} \\[3pt] \hline
			\endhead
			
			\hline 
			\endlastfoot
			
			0.1 & Napravljen predložak.	& Sokolić & 16.10.2020. 		\\[3pt] \hline 
			0.2	& Dodan opis projektnog zadatka & Gojević &  20.10.2020.	\\[3pt] \hline 
			0.5 & Dodan \textit{Use Case} dijagram i jedan sekvencijski dijagram, funkcionalni i nefunkcionalni zahtjevi i dodatak A &  &  \\[3pt] \hline 
			0.6 & Arhitektura i dizajn sustava, algoritmi i strukture podataka &  &  \\[3pt] \hline 
			0.8 & Povijest rada i trenutni status implementacije,\newline Zaključci i plan daljnjeg rada &  &  \\[3pt] \hline 
			0.9 & Opisi obrazaca uporabe &  &  \\[3pt] \hline 
			0.10 & Preveden uvod &  &  \\[3pt] \hline 
			0.11 & Sekvencijski dijagrami &  &  \\[3pt] \hline 
			0.12.1 & Započeo dijagrame razreda &  &  \\[3pt] \hline 
			0.12.2 & Nastavak dijagrama razreda &  &  \\[3pt] \hline 
			\textbf{1.0} & Verzija samo s bitnim dijelovima za 1. ciklus &  &  \\[3pt] \hline 
			1.1 & Uređivanje teksta -- funkcionalni i nefunkcionalni zahtjevi &  &  \\[3pt] \hline 
			1.2 & Manje izmjene:Timer - Brojilo vremena &  &  \\[3pt] \hline 
			1.3 & Popravljeni dijagrami obrazaca uporabe &  &  \\[3pt] \hline 
			1.5 & Generalna revizija strukture dokumenta &  &  \\[3pt] \hline 
			1.5.1 & Manja revizija (dijagram razmještaja) &  &  \\[3pt] \hline 
			\textbf{2.0} & Konačni tekst predloška dokumentacije  &  &  \\[3pt] \hline 
			&  &  & \\[3pt] \hline
			
			
		\end{longtabu}
	
	
		\textit{Moraju postojati glavne revizije dokumenata 1.0 i 2.0 na kraju prvog i drugog ciklusa. Između tih revizija mogu postojati manje revizije već prema tome kako se dokument bude nadopunjavao. Očekuje se da nakon svake značajnije promjene (dodatka, izmjene, uklanjanja dijelova teksta i popratnih grafičkih sadržaja) dokumenta se to zabilježi kao revizija. Npr., revizije unutar prvog ciklusa će imati oznake 0.1, 0.2, …, 0.9, 0.10, 0.11.. sve do konačne revizije prvog ciklusa 1.0. U drugom ciklusu se nastavlja s revizijama 1.1, 1.2, itd.}