\chapter{Arhitektura i dizajn sustava}
		
		\textbf{\textit{dio 1. revizije}}\\

		\textit{ Potrebno je opisati stil arhitekture te identificirati: podsustave, preslikavanje na radnu platformu, spremišta podataka, mrežne protokole, globalni upravljački tok i sklopovsko-programske zahtjeve. Po točkama razraditi i popratiti odgovarajućim skicama:}
	\begin{itemize}
		\item 	\textit{izbor arhitekture temeljem principa oblikovanja pokazanih na predavanjima (objasniti zašto ste baš odabrali takvu arhitekturu)}
		\item 	\textit{organizaciju sustava s najviše razine apstrakcije (npr. klijent-poslužitelj, baza podataka, datotečni sustav, grafičko sučelje)}
		\item 	\textit{organizaciju aplikacije (npr. slojevi frontend i backend, MVC arhitektura) }		
	\end{itemize}

	
		

		

				
		\section{Baza podataka}
			
			\text Za potrebe našeg sustava koristit ćemo NoSQL baza podataka orijentirana na dokumente, odnosno kolekcije dokumenata. Gradivna jedinka baze je dokument koji je definiran svojim imenom i skupom atributa. Zadaća baze podataka jest jednostavna i brza pohrana, izmjena i dohvat podataka za daljnju obranu. Baza podataka ove aplikacije (za sada) sastoji se od jednog entiteta, a to je:
			
			\begin{itemize}
				\item Korisnik
			\end{itemize}
		
			\subsection{Opis tablica}
			

				\textbf{Korisnik} \text    Ovaj entitet sadržava sve važne informacije o korisniku aplikacije. Sadrži atribute: about, cardExp, creditCard, firstName, iban, image, lastName, lecturer, mail, secCode, username. Za svakog novog registriranog korisnika kreira se dokument pod šifrom u bazi podataka.
				
				\begin{longtabu} to \textwidth {|X[6, l]|X[6, l]|X[20, l]|}
					
					\hline \multicolumn{3}{|c|}{\textbf{Korisnik}}	 \\[3pt] \hline
					\endfirsthead
					
					\hline \multicolumn{3}{|c|}{\textbf{Korisnik}}	 \\[3pt] \hline
					\endhead
					
					\hline 
					\endlastfoot
					
					About & string & kratka biografija o korisniku ukoliko je predavač \\ \hline
					CardExp & string & datum isteka kartice za naplatu \\ \hline
					CreditCard & string & broj kartice za naplatu \\ \hline
					FirstName & string & ime korisnika \\ \hline
					iban & string & IBAN računa za isplatu honorara \\ \hline
					Image & & fotografija korisnika ukoliko je predavač \\ \hline
					LastName & string & prezime korisnika \\ \hline
					Lecturer & boolean & oznaka je li korisnik registriran kao predavač \\ \hline
					Mail & string & mail korisnika \\ \hline
					SecCode & string & kod za verifikaciju kartice \\ \hline
					Username & string & korisničko ime \\ \hline
					
					
				\end{longtabu}
			
			
			\subsection{Dijagram baze podataka}
				
			
			\eject
			
			
		\section{Dijagram razreda}
		
			\textit{Potrebno je priložiti dijagram razreda s pripadajućim opisom. Zbog preglednosti je moguće dijagram razlomiti na više njih, ali moraju biti grupirani prema sličnim razinama apstrakcije i srodnim funkcionalnostima.}\\
			
			\textbf{\textit{dio 1. revizije}}\\
			
			\textit{Prilikom prve predaje projekta, potrebno je priložiti potpuno razrađen dijagram razreda vezan uz \textbf{generičku funkcionalnost} sustava. Ostale funkcionalnosti trebaju biti idejno razrađene u dijagramu sa sljedećim komponentama: nazivi razreda, nazivi metoda i vrste pristupa metodama (npr. javni, zaštićeni), nazivi atributa razreda, veze i odnosi između razreda.}\\
			
			\textbf{\textit{dio 2. revizije}}\\			
			
			\textit{Prilikom druge predaje projekta dijagram razreda i opisi moraju odgovarati stvarnom stanju implementacije}
			
			
			
			\eject
		
		\section{Dijagram stanja}
			
			
			\textbf{\textit{dio 2. revizije}}\\
			
			\textit{Potrebno je priložiti dijagram stanja i opisati ga. Dovoljan je jedan dijagram stanja koji prikazuje \textbf{značajan dio funkcionalnosti} sustava. Na primjer, stanja korisničkog sučelja i tijek korištenja neke ključne funkcionalnosti jesu značajan dio sustava, a registracija i prijava nisu. }
			
			
			\eject 
		
		\section{Dijagram aktivnosti}
			
			\textbf{\textit{dio 2. revizije}}\\
			
			 \textit{Potrebno je priložiti dijagram aktivnosti s pripadajućim opisom. Dijagram aktivnosti treba prikazivati značajan dio sustava.}
			
			\eject
		\section{Dijagram komponenti}
		
			\textbf{\textit{dio 2. revizije}}\\
		
			 \textit{Potrebno je priložiti dijagram komponenti s pripadajućim opisom. Dijagram komponenti treba prikazivati strukturu cijele aplikacije.}