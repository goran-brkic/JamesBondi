\chapter{Implementacija i korisničko sučelje}
		
		
		\section{Korištene tehnologije i alati}
		
			\text Komunikacija u timu realizirana je korištenjem aplikacija WhatsApp\footnote{\url{https://www.whatsapp.com/}} i Messenger za Facebook\footnote{\url{https://https://www.messenger.com/}}, a sastanci su održavani putem platformi Microsoft Teams\footnote{\url{https://www.microsoft.com/hr-hr/microsoft-365/microsoft-teams/}} te Discord\footnote{url{https://discord.com/}}. Za izradu UML dijagrama korišten je alat Astah UML\footnote{\url{https://astah.net/products/astah-uml/}}, a kao sustav za upravljanje izvornim kodom Git\footnote{\url{https://git-scm.com/}}. Udaljeni repozitorij projekta dostupan je na web platformi GitLab\footnote{\url{https://gitlab.com/}}.
			
			Kao razvojno okruženje korišten je Visual Studio Code\footnote{\url{https://code.visualstudio.com/}} – uređivač koda tvrtke Microsoft. Dostupan je za Window, Linux i macOS. Uključuje podršku za otklanjanje pogrešaka, isticanje sintakse, inteligentno dovršavanje koda, refaktoriranje koda i ugrađeni Git. Inicijalno ima ugrađenu podršku za JavaScript, TypeScript i Node.js, no ima bogat sustav ekstenzija za druge jezike (na primjer C++, Java, Phyton, Dart) i druge radne okvire (.NET, Unity…)
			
			Aplikacija je napisana korištenjem softera Flutter\footnote{\url{https://flutter.dev/}} i programskog jezika Dart\footnote{\url{https://dart.dev/}}. Flutter je softver za razvoj korisničkog sučelja otvorenog koda, razvijen od tvrtke Google. Omogućava višeplatformski razvoj korištenjem jednog koda te jedostavnost izrade aplikacija. Flutter koristi programski jezik Dart, kojim je realiziran i frontend i backend.
			
			Baza podataka nalazi se na poslužitelju u oblaku Firestore\footnote{\url{https://firebase.google.com/docs/firestore}}. 
			
			
			
			\eject 
		
	
		\section{Ispitivanje programskog rješenja}
			
			\textbf{\textit{dio 2. revizije}}\\
			
			 \textit{U ovom poglavlju je potrebno opisati provedbu ispitivanja implementiranih funkcionalnosti na razini komponenti i na razini cijelog sustava s prikazom odabranih ispitnih slučajeva. Studenti trebaju ispitati temeljnu funkcionalnost i rubne uvjete.}
	
			
			\subsection{Ispitivanje komponenti}
			\textit{Potrebno je provesti ispitivanje jedinica (engl. unit testing) nad razredima koji implementiraju temeljne funkcionalnosti. Razraditi \textbf{minimalno 6 ispitnih slučajeva} u kojima će se ispitati redovni slučajevi, rubni uvjeti te izazivanje pogreške (engl. exception throwing). Poželjno je stvoriti i ispitni slučaj koji koristi funkcionalnosti koje nisu implementirane. Potrebno je priložiti izvorni kôd svih ispitnih slučajeva te prikaz rezultata izvođenja ispita u razvojnom okruženju (prolaz/pad ispita). }
			
			
			
			\subsection{Ispitivanje sustava}
			
			 \textit{Potrebno je provesti i opisati ispitivanje sustava koristeći radni okvir Selenium\footnote{\url{https://www.seleniumhq.org/}}. Razraditi \textbf{minimalno 4 ispitna slučaja} u kojima će se ispitati redovni slučajevi, rubni uvjeti te poziv funkcionalnosti koja nije implementirana/izaziva pogrešku kako bi se vidjelo na koji način sustav reagira kada nešto nije u potpunosti ostvareno. Ispitni slučaj se treba sastojati od ulaza (npr. korisničko ime i lozinka), očekivanog izlaza ili rezultata, koraka ispitivanja i dobivenog izlaza ili rezultata.\\ }
			 
			 \textit{Izradu ispitnih slučajeva pomoću radnog okvira Selenium moguće je provesti pomoću jednog od sljedeća dva alata:}
			 \begin{itemize}
			 	\item \textit{dodatak za preglednik \textbf{Selenium IDE} - snimanje korisnikovih akcija radi automatskog ponavljanja ispita	}
			 	\item \textit{\textbf{Selenium WebDriver} - podrška za pisanje ispita u jezicima Java, C\#, PHP koristeći posebno programsko sučelje.}
			 \end{itemize}
		 	\textit{Detalji o korištenju alata Selenium bit će prikazani na posebnom predavanju tijekom semestra.}
			
			\eject 
		
		
		\section{Dijagram razmještaja}
			
			Dijagram razmještaja opisuje topologiju sustava i usredotočen je na odnos sklopovskih i programskih dijelova. Na slici \ref{fig:Dijagram_razmjestaja} prikazan je specifikacijski dijagram razmještaja koji prikazuje komunikaciju mobilnog uređaja s operacijskim sustavom Android na kojem se nalazi mobilna aplikacija s bazom podataka koja se nalazi u oblaku. Mobilna aplikacija spaja se HTTP protokolom na oblak. U oblaku se unutar Firebase platforme nalazi baza podataka Cloud Firestore, usluga za pohranu podataka Storage i Firebase Authentication za određivanje identiteta korisnika.
			
			
			\begin{figure}[h]
				\includegraphics[scale=0.6]{dijagrami/Dijagram_razmjestaja.PNG}
				\centering
				\caption{Dijagram razmještaja}
				\label{fig:Dijagram_razmjestaja}
			\end{figure}
		
			
			\eject 
		
		\section{Upute za puštanje u pogon}
		
			\textbf{\textit{dio 2. revizije}}\\
		
			 \textit{U ovom poglavlju potrebno je dati upute za puštanje u pogon (engl. deployment) ostvarene aplikacije. Na primjer, za web aplikacije, opisati postupak kojim se od izvornog kôda dolazi do potpuno postavljene baze podataka i poslužitelja koji odgovara na upite korisnika. Za mobilnu aplikaciju, postupak kojim se aplikacija izgradi, te postavi na neku od trgovina. Za stolnu (engl. desktop) aplikaciju, postupak kojim se aplikacija instalira na računalo. Ukoliko mobilne i stolne aplikacije komuniciraju s poslužiteljem i/ili bazom podataka, opisati i postupak njihovog postavljanja. Pri izradi uputa preporučuje se \textbf{naglasiti korake instalacije uporabom natuknica} te koristiti što je više moguće \textbf{slike ekrana} (engl. screenshots) kako bi upute bile jasne i jednostavne za slijediti.}
			
			
			 \textit{Dovršenu aplikaciju potrebno je pokrenuti na javno dostupnom poslužitelju. Studentima se preporuča korištenje neke od sljedećih besplatnih usluga: \href{https://aws.amazon.com/}{Amazon AWS}, \href{https://azure.microsoft.com/en-us/}{Microsoft Azure} ili \href{https://www.heroku.com/}{Heroku}. Mobilne aplikacije trebaju biti objavljene na F-Droid, Google Play ili Amazon App trgovini.}
			
			
			\eject 